\usepackage[english]{babel}                          % deutsche Trennmuster
% \usepackage{ae}                           % beseitigt Fehler beim pdf erzeugen

\usepackage[T1]{fontenc}                    % T1-kodierte Schriften, korrekte Trennmuster fuer Worte mit Umlauten
\usepackage[utf8]{inputenc}
\usepackage[default]{opensans}
\usepackage{courier}                        % Courier\ttdefault
\usepackage{microtype}
\usepackage{mathswap}

\usepackage{amsmath,amsthm,amsfonts,amssymb}
\usepackage{esint}                          %schöne integrale, benötigt amsmath

\usepackage{wrapfig}                        % text um bilder fliesen lassen
\usepackage{longtable}
\usepackage{booktabs}
\usepackage{color}                          % Schriftfarbe
\usepackage{lscape}
\usepackage{subcaption}
\usepackage[subfigure]{tocloft}
\usepackage{appendix}
\usepackage{caption}
\usepackage{graphicx}
\usepackage{setspace}
\usepackage{pdfpages}
\usepackage{verbatim}
\usepackage[tight,ugly]{units}
% \usepackage{lmodern}
\usepackage[hidelinks,
pdftitle={Coupling of an artificial neural network with LES-LBM to imrpove wind farm control},
pdfsubject={Diploma thesis on coupling an artificial neural network with LES-LBM to improve wind farm control},
pdfauthor={Henry Korb},
pdfkeywords={Diplomathesis, LBM, Reinforcement Learning, Wind farm control, actuator line}]{hyperref}              %links an referenzen setzen
\usepackage{pgfplots}
\pgfplotsset{compat=1.15}

\usepackage{siunitx}
\usepackage{float}
\usepackage{placeins}
% packages für Programmablaufplan
\usepackage{tikz}
\usetikzlibrary{shapes,arrows}
\usepgflibrary{arrows}


\usepackage{pgf-umlsd}
\usepackage{algorithm}
\usepackage{algorithmic}
\DeclareMathOperator*{\argmax}{arg\,max}


%plus(minus)
% \usepackage{relsize}
% \newcommand*\myPM{\ensuremath{\substack{+\\[-0.25em]-}\,}}
% \newcommand*\myPMbM{\ensuremath{\substack{+\\[-0.25em]\mathsmaller(-\mathsmaller)}\,}}
% \newcommand*\myPMbP{\ensuremath{\substack{\mathsmaller(+\mathsmaller)\\[-0.25em]-}\,}}

% defninierte Farben
\definecolor{fond}{RGB}{240,240,240}

\renewcommand{\vec}[1]{\boldsymbol{\mathrm{#1}}}
\renewcommand{\b}[1]{\mathbf{#1}}
\newcommand{\its}{\it\sffamily}
\renewcommand{\captionlabelfont}{\bfseries\sffamily}                                                          % Bildunterschriften fett, sf-Schriftart
\renewcommand{\captionfont}{\sffamily}                                                                          % Bildunterschriften sf-Schriftart
\captionsetup[table]{singlelinecheck = true, aboveskip=0pt, belowskip=7pt}  % Format Tabellenüberschriften
\newcommand{\tabformat}{\small\sffamily}                                                                        % Serifenlose und kleine Schrift in Tabelle
\renewcommand{\arraystretch}{1.4}                                                                                       % Groessere Abstaende zwischen Zeilen in Tabelle

% Seitenaufbau
\setstretch{1.4}                % Zeilenabstand
\setlength{\parskip}{6pt}       % Extra Abstand bei Absatz mit \par - Befehl
\setlength{\parindent}{0pt}     % kein Einrücken des Textes in erster Zeile eines Absatzes
\raggedbottom                   %seite nicht zwingend bis unten vollsetzen

% Seitenlayout, Seitenkopf und Fuss gestalten
\setlength{\topmargin}{-15mm}       % Abstand des Kopfes zu der Seitenoberkante

\usepackage[outer=20mm,inner=30mm,bottom=40mm,headsep=10mm,footskip=12mm]{geometry}
% \addtokomafont{pagehead}{\scshape}
% \pagestyle{headings}

%kopf- und fußzeilen
%\renewcommand{\familydefault}{\sfdefault}   %serifenlose schrift
\usepackage[headsepline]{scrlayer-scrpage}
\automark[chapter]{chapter}
%\automark*{section}
\clearpairofpagestyles
\ohead{\headmark}
\ofoot*{\pagemark}


% Punkt + Komma Abstände bei Tausendern/Dezimalzahlen ans dt. anpassen
%\mathcode`,="013B
%\mathcode`.="613A

% Literaturverzeichnis
%\usepackage{cite}
\usepackage[backend=biber, style=trad-unsrt, giveninits=true, doi=false, isbn=false, eprint=false, url=false, arxiv=pdf, sorting=nyt, maxnames=3, date=year]{biblatex}
\usepackage{csquotes}
%\usepackage{bibgerm}
\renewcommand{\pnumfont}{\sffamily}                     % Seitenzahlen serifenlos
\renewcommand{\cftsecpagefont}{\sffamily}
\renewcommand{\cftsubsecpagefont}{\sffamily}
\renewcommand{\cftsubsecindent}{1.5em}
\renewcommand{\cftsecfont}{\sffamily}
\renewcommand{\cftsubsecfont}{\sffamily}
\renewcommand{\cftchapdotsep}{4.1}
\renewcommand*{\chapterheadstartvskip}{\vspace*{-1cm}}

\renewcommand{\equationautorefname}{Eq.}
\usepackage{pdflscape}
\def\algorithmautorefname{Algorithm}
\usepackage{float}


\newcommand{\legendOne}[1]{{\color[HTML]{1E90FF} \rule[3pt]{22pt}{1pt}} #1 }
\newcommand{\legendTwo}[2]{\legendOne{#1}{\color[HTML]{DC143C}  \rule[3pt]{5pt}{1pt} \rule[3pt]{5pt}{1pt} \rule[3pt]{5pt}{1pt}} #2 }
\newcommand{\legendThree}[3]{\legendTwo{#1}{#2}{\color[HTML]{228B22} \rule[3pt]{5pt}{1pt} \rule[3pt]{1pt}{1pt} \rule[3pt]{5pt}{1pt} \rule[3pt]{1pt}{1pt}} #3 }
\newcommand{\legendFour}[4]{\legendThree{#1}{#2}{#3}{\color[HTML]{000080} \rule[3pt]{3pt}{1pt} \rule[3pt]{3pt}{1pt} \rule[3pt]{3pt}{1pt} \rule[3pt]{3pt}{1pt}} #4 }
\newcommand{\legendFive}[5]{\legendFour{#1}{#2}{#3}{#4}{\color{black} \rule[3pt]{1pt}{1pt} \rule[3pt]{1pt}{1pt} \rule[3pt]{1pt}{1pt} \rule[3pt]{1pt}{1pt} \rule[3pt]{1pt}{1pt} \rule[3pt]{1pt}{1pt}} #5 }