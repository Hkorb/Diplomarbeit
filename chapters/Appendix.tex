% !TEX root = ../Diploma.tex
\chapter{The Cumulant Lattice Boltzmann Method}
\label{app:cumulant}
\section{Derivation of cumulants}
To provide a more detailed explanation of the cumulant LBM, a more thorough derivation is given here. Note that in the following, the linear index $i$ of the discrete populations is split into three indeces $i$, $j$, $k$ representing the lattice directions, running from $-1$ to $1$. First, the discrete distribution is written in a continuous form and transformed to frequency space, making it indepent of the frame of reference:
\begin{align}
f(\vec{\xi}) &= \sum_{i,j,k} f_{i,j,k} \delta(ic_s - \xi) \delta(jc_s - \upsilon) \delta(kc_s - \zeta) \label{eq:contin}\\
F(\vec{\Xi}) &= \int_{-\infty}^{\infty}f(\vec{\xi})e^{-\vec{\Xi} \cdot \vec{\xi}} \mathrm{d}\vec{\xi}, \label{eq:cum_laplace}
\end{align}
with $\vec{\Xi} = [\Xi, \Upsilon, \mathrm{Z}]^T$ as the frequency vector. This distribution is then used in the cumulant generating function:
\begin{equation}
	C_{\alpha,\beta,\gamma} = c_s^{-(\alpha+\beta+\gamma)} \left. \frac{\partial^\alpha \partial^\beta \partial^\gamma}{\partial \Xi^\alpha \partial \Upsilon^\beta \partial \mathrm{Z}^\gamma} \ln F(\Xi, \Upsilon, \mathrm{Z})\right|_{\Xi=\Upsilon=\mathrm{Z}=0}. \label{eq:cum}
\end{equation} However, from that form, the cumulants are not directly computable. Comparing the cumulant generating function to the moment generating function
\begin{equation}
	M_{\alpha, \beta, \gamma} = c_s^{-(\alpha+\beta+\gamma)} \left. \frac{\partial^\alpha \partial^\beta \partial^\gamma}{\partial \Xi^\alpha \partial \Upsilon^\beta \partial \mathrm{Z}^\gamma} F(\Xi, \Upsilon, \mathrm{Z}) \right|_{\Xi=\Upsilon=\mathrm{Z}=0} = \sum_{i,j,k} i^\alpha j^\beta k^\gamma f_{i,j,k} \label{eq:moments}
\end{equation} shows their similarity. Computing the cumulants of an arbitrary function shows how they can be expressed in terms of moments, two examples are given here:
\begin{align}
	C_{200} &= \frac{M_{200}}{M_{000}} - \frac{M_{100}^2}{M_{000}^2} \label{eq:c_200}\\
	C_{110} &= \frac{M_{110}}{M_{000}} - \frac{M_{100} M_{010}}{M_{000}^2}. \label{eq:c_110}
\end{align}
Cumulants can also be computed from central moments, which is requires less computations \cite{geier_cumulant_2015}. Cumulants of zeroth and first order stay constant throughout the collision. The cumulants upwards from order two can now be relaxed indivdually according to
\begin{equation}
	C_{\alpha, \beta, \gamma}^* = \left(1-\hat{\omega}_{\alpha,\beta,\gamma}\right) C_{\alpha, \beta, \gamma}, \label{eq:relax}
\end{equation}
therefore each has its own relaxation frequency. The first frequency $\hat{\omega}_1$ governs the shear viscosity and $\hat{\omega}_2$ the bulk viscosity of the fluid by the same relation as \eqref{eq:nu}. The rest of the frequencies can be chosen freely and are usually set to one, resulting fully relaxing the cumulants, however a parametrized version of the collision operation exists, that changes some of the higher order relaxation frequencies, leading to fourth order accurate diffusion in a fully resolved simulation \cite{geier_parametrization_2017}. In underresolved simulations the parameter introduced in the parametrization can be used to influence the diffusivity and therefore acting as an implicit subgrid scale model, however so far only empirical evidence exists for this.\cite{geier_cumulant_2015}
The density, velocity and the terms of velocity gradient tensor can be identified with cumulants:
\begin{align}
\delta \rho &= M_{000} = C_{000} \label{eq:drho}\\
u &= \frac{M_{100}}{M_{000}} =  C_{100}\\
D_x v + D_y u &= -3 \hat{\omega}_1 C_{110} = k_{xy} \label{eq:derivate1}\\
D_x u &= - \frac{\hat{\omega}_1}{2 \rho}\left(2 C_{200} - C_{020} - C_{002} \right) - \frac{\hat{\omega}_2}{2 \rho} \left(C_{200} + C_{020} + C_{002} - \delta \rho \right) \\
D_x u - D_y v &= - \frac{3\hat{\omega}_1}{2} \left(C_{200} - C_{020}\right) = k_{xx-yy}. \label{eq:derivative2}
\end{align} Note, that in this derivation the well-conditioned cumulant is used as described in the appendix of \cite{geier_cumulant_2015} and therefore the zeroth order cumulant is not the density but its deviation from unit density $\delta \rho$. Also the additional moments $k_{xy}$ and $k_{xx-yy}$ are introduced for use in \autoref{app:refinement}.
\section{Refinement}
\label{app:refinement}
As was stated before, LBM is usually used on a uniform grid. However, often a variation in grid spacing is desirable, as coarser grids need less memory and have a coarser timestep, whereas a finer grid is able to resolve finer turbulent structures. To balance these two interests, the domain can be partitioned into blocks with different levels of refinement, as areas of high turbulence, which need high resolution, usually occur only in known parts of the domain. The question is now how to treat the borders of the blocks. In \cite{schonherr_towards_2015} the compact interpolation scheme is proposed that is second order accurate, therefore it is consistent with the order of accuracy of the LBM in general. However, the derivation given was found to be neither exhaustive, nor fully correct, as there seem to be misprints, making it difficult to follow. Therefore a more detailed derivation is given here, that is based on \cite{schonherr_towards_2015} as well as \cite{kutscher_multiscale_2019}, which gave a version for an incompressible form of LBM, but suffers from similar issues. In each coarse timestep, a layer of coarse nodes is interpolated from fine nodes and two layers of fine nodes are interpolated from coarse nodes. The layers overlap, so that at no point in time invalid populations are propagated outside the overlap \cite{schonherr_towards_2015}. The node being interpolated is called the receiver node while the nodes from which is interpolated is referred to as the donor node. Each receiver node has eight donor nodes. A local coordinate system in the center of cube is defined, with the donor nodes laying in the corners at $x={-0.5, 0.5}$, $y={-0.5,0.5}$ and $z={-0.5,0.5}$. \\First, a interpolation function for the density, $\delta \hat{\rho}$ and the velocities $\hat{u}$, $\hat{v}$ and $\hat{w}$ is defined. Note that the density only requires to be first order accurate:
\begin{align}
	 \delta \hat{\rho} &= d_0 + d_x x + d_y y + d_z z + d_{xy}xy + d_{xz}xz + d_{yz}yz + d_{xyz}xyz \\
	\hat{u} &= a_0 + a_x x + a_y y + a_z z + a_{xx} x^2 + a_{xy} xy + a_{xz} xz + a_{yy} y^2 + a_{yz} yz + a_{zz} z^2 + a_{xyz}xyz \\
	\hat{v} &= b_0 + b_x x + b_y y + b_z z + b_{xx} x^2 + b_{xy} xy + b_{xz} xz + b_{yy} y^2 + b_{yz} yz + b_{zz} z^2 + b_{xyz}xyz \\
	\hat{w} &= c_0 + c_x x + c_y y + c_z z + c_{xx} x^2 + c_{xy} xy + c_{xz} xz + c_{yy} y^2 + c_{yz} yz + c_{zz} z^2 + c_{xyz}xyz
\end{align}
To evualate the interpolation functions, constraints have to be defined. The velocity and density in the donor node place 32 constraints, since there are eight of each, thus nine more are required for the 41 degrees of freedom. The second derivative of velocities in the center of the local coordinate system are chosen as these additional constraints. They can be computed by taking the first order central difference of the terms of the velocity gradient tensor, which will be denoted as $D_xx u$ and so on. However, the terms in the main diagonal of the velocity gradient tensor are lengthy and include $\hat{\omega}_2$. This is somewhat undesirable and it was shown in \eqref{eq:derivate1} that differences of the terms can be expressed more directly in differences of cumulants. Therefore the remaining nine constraints are chosen like this:
\begin{align}
\frac{\partial^2}{\partial x^2} \hat{v}+\frac{\partial^2}{\partial y \partial x} \hat{u} &= D_{xx} v + D_{xy} u \label{eq:const1} \\
2\frac{\partial^2}{\partial x^2} \hat{u} - \frac{\partial^2}{\partial x \partial y} \hat{v} - \frac{\partial^2}{\partial x \partial z} \hat{w} &= 2 D_{xx}u- D_{xy}v - D_{xz}w \\
2\frac{\partial^2}{\partial y^2} \hat{v} - \frac{\partial^2}{\partial x \partial y} \hat{u} - \frac{\partial^2}{\partial y \partial z} \hat{w} &= 2 D_{yy}v- D_{xy}u - D_{yz}w \\
2\frac{\partial^2}{\partial z^2} \hat{w} - \frac{\partial^2}{\partial y \partial z} \hat{v} - \frac{\partial^2}{\partial x \partial z} \hat{u} &= 2 D_{zz}w- D_{yz}v - D_{xz}u, \label{eq:const2}
\end{align}
the missing five constraints by the off diagonal can easily be extrapolated. \\
Thus a system of linear equations is established that can be solved for the coefficients of the interpolation functions:
\begin{align}
a_0 &= \frac{1}{32} \sum_{x,y,z}  -4 D_{xx}u-2 D_{xy}v -4D_{yy}u- 2 D_{xz}w + -4 D_{zz}u + 4 u_{xyz} + 4xyv_{xyz} + 4 xzw_{xyz} \\
&= \frac{1}{32} \sum_{x,y,z}  -x\left(k_{xx-yy} + k_{xx-zz}\right) - 2yk_{xy} -2zk_{xz} + 4 u_{xyz} + 4xyv_{xyz} + 4 xzw_{xyz} \\
a_x &= \frac{1}{2} \sum_{x,y,z} x u_{xyz} \\
a_{xx} &= \frac{1}{8} \sum_{x,y,z}x\left(k_{xx-yy}+k_{xx-zz}\right) + 4xyv_{xyz}+ 4 xzw_{xyz} \\
a_{yy} &= \frac{1}{8} \sum_{x,y,z}y k_{xy} - 4 xy v_{xyz} \\
a_{xy} &= 2 \sum_{x,y,z} xyu_{xyz} \\
a_{xyz} &= 8 \sum_{x,y,z} xyzu_{xyz} \\
d_0 &= \frac{1}{8} \sum_{x,y,z} \delta \rho_{xyz},
\end{align}
in order to reduce the number of equations only unique descriptions are given, the other coefficients can easily extrapolated. Furthermore note that the coefficients of $\delta \hat{\rho}$ are the same as for the other interpolation functions with the exception of $d_0$.
Thus the velocities are computed to second order. To arrive at equations for the second order cumulants, \eqref{eq:derivate1} - \eqref{eq:derivative2} are solved for cumulants. A term including the divergence of the velocity is neglected, since LBM is valid in a weakly compressible range and the term is thus much smaller.To compute the cumulants, the derivatves of $\hat{u}$, $\hat{v}$ and $\hat{w}$ are used and the constant terms in the derivatives are replaced with averaged values of second order moments. Also $\hat{\omega}_1$ has to be scaled with a factor $\sigma_{rf}$ to account for the change in refinement, it is $2$ when scaling from fine to coarse and $1/2$ when scaling from coarse to fine:
\begin{align}
	A_{011} &= \frac{\partial \hat{v}}{\partial z} + \frac{\partial \hat{w}}{\partial y} - b_z - c_y \\
	&= b_{xz} x + b_{yz} y + 2 b_{zz} z + b_{xyz} xy + c_{xy} x + 2 c_{yy} y + c_{yz} z + c_{xyz} xz \\
	A_{101} &= \frac{\partial \hat{u}}{\partial z} + \frac{\partial \hat{w}}{\partial x} - a_z - c_x \\
	&= a_{xz} x + a_{yz} y + 2 a_{zz} z + a_{xyz} xy + 2 c_{xx} x + c_{xy} y + c_{xz} z + c_{xyz} yz \\
	A_{110} &= \frac{\partial \hat{u}}{\partial y} + \frac{\partial \hat{v}}{\partial x} - a_y - b_x \\
	&= a_{xy} x + 2 a_{yy} y + a_{yz} z + a_{xyz} xz + 2 b_{xx} x + b_{xy} y + b_{xz} z + b_{xyz} yz \\
	B &= \frac{\partial \hat{u}}{\partial x} - \frac{\partial \hat{v}}{\partial y} - a_x + b_y \\
	&= 2 a_{xx} x + a_{xy} y + a_{xz} z + a_{xyz} yz - b_{xy} x - 2 b_{yy} y - b_{yz} z - b_{xyz} xz \\
	C &= \frac{\partial \hat{u}}{\partial x} - \frac{\partial \hat{w}}{\partial z} - a_x + c_z \\
	&= 2 a_{xx} x + a_{xy} y + a_{xz} z + a_{xyz} yz - c_{xz} x - c_{yz} y - 2 c_{zz} y - c_{xyz} xy\\
	C_{011} &= - \frac{\sigma_{rf} \rho}{3 \hat{\omega}_d} \left( \overline{k_{yz}} + A_{011} \right)\\
	C_{101} &= - \frac{\sigma_{rf} \rho}{3 \hat{\omega}_d} \left( \overline{k_{xz}} + A_{101} \right) \\
	C_{110} &= - \frac{\sigma_{rf} \rho}{3 \hat{\omega}_d} \left( \overline{k_{xy}} + A_{110} \right) \\
	C_{200} &= \frac{\delta \rho}{9} - \frac{2 \sigma_{rf} \rho}{9 \hat{\omega}_d} \left( \overline{k_{xx-yy}} + B + \overline{k_{xx-zz}} + C \right) \\
	C_{020} &= \frac{\delta \rho}{9} - \frac{2 \sigma_{rf} \rho}{9 \hat{\omega}_d} \left( -2 \left(\overline{k_{xx-yy}} + B\right) + \overline{k_{xx-zz}} + C \right) \\
	C_{002} &= \frac{\delta \rho}{9} - \frac{2 \sigma_{rf} \rho}{9 \hat{\omega}_d} \left( \overline{k_{xx-yy}} + B - 2 \left( \overline{k_{xx-zz}} + C \right) \right).
\end{align}
Now the cumulants can be transformed back to distributions, with the central moments up to order two and cumulants of order higher than two set to zero.
