\chapter*{Kurzfassung}
\thispagestyle{empty}
\vspace{0cm}

\textbf{\large Validierung eines Wandmodels für Large Eddy Simulationen auf Basis der Lattice-Boltzmann Methode}\\


In dieser Arbeit wurde eine Sammlung von Testfällen zusammengestellt, die darauf ausgelegt ist, das Wandmodell einer wandmodellierten Large Eddy Simulation mit der Lattice Boltzmann Methode zu untersuchen. Unter den Testfällen sind die turbulente Kanalströmung, mit zusätzlichen Modifikationen der Geometrie um auch den Einfluss des Wandabstands und Winkels zum Gitter zu untersuchen. Zusätzlich gehören auch die Kanalströmung mit geschlossenen Seiten, die turbulente Couette-Strömung und der periodische Hügel zu den Testfällen. Zudem wurde das Wandmodel im LBM-Code ultraFluidX™ der Altair Engineering GmbH mit diesen Testfällen untersucht. Es wurde gezeigt, dass das Wandmodell nur einen geringen, aber meist negativen Einfluss auf die mittleren Geschwindigkeit nahe der Wand hat, jedoch zu deutlichen Verbesserungen bei den Werten der Geschwindigkeitsfluktuationen führt. Ausserdem wurde herausgefunden, dass der Einfluss des Wandmodells mit steigender Reynoldszahl und gröberer Auflösung zunimmt. Zudem reagierte das Wandmodell sensitiv auf die Inklination und den Wandabstand zum Gitter.
\vspace{2cm}  \\
{\bfseries \sffamily \huge Abstract}
\vspace{1cm}

\textbf{\large Validation of a wall model for Large Eddy Simulations based on the Lattice Boltzmann Method}\\

In this work, a collection of test cases was assembled, that is suited to analyze the wall model of wall modeled Large Eddy Simulation of Lattice Boltzmann method. The test cases include a turbulent Channel Flow, with additional modifications of geometry, to also examine the influence the distance and the angle to the lattice have. Furthermore, a Duct Flow, a turbulent Couette Flow and the Periodic Hill are included. Additionally, the wall model implemented in the LBM code ultraFluidX™ was analyzed with the test case collection. It was found that the application of the wall model has a small negative effect on the mean velocity in the vicinity of the wall, but improves the results for the velocity fluctuations significantly. It was also found, that the influence of the wall model increases with Reynolds number and coarser resolution. Furthermore, the wall models results were sensitive to inclination and distance from wall to node.\\