While human made climate change only begins to effect the countries of Europe and North America, the predictions for the near and far future are devastating \cite{hoegh-guldberg_impacts_2019}. One of the key causes for global warming is increase in atmospheric greenhouse gases like carbondioxide and methane. Electricity production accounts  up to a third of greenhouse gas emissions in major industrial countries like the United States \cite{hockstad_inventory_2018} and Germany \cite{ortl_entwicklung_2020}. Therefore, the energy sector is shifting to renewable sources of energy such as wind energy \cite{international_energy_agency_global_2020}\\
The momentum of the wind is used to power wind turbines, which generate electricity. To reduce investment and maintenance costs, wind turbines are often arranged in wind farms of multiple, sometimes hundreds of turbines. However, this also reduces the overall efficiency of the turbines due to wake losses. These losses can be mitigated by various means. Placing turbines further apart reduces the wake interaction but also some of the advantages of wind farms. Furthermore, this can not be changed for farms already built. A different approach is to change the controller of turbines in order to reduce wake interaction. This has the advantage that it is not only applicable to newly built parks, but can also be implemented at already existing sites. \\
One possible way to reduce wake interaction is by curtailing the upstream turbines. The 