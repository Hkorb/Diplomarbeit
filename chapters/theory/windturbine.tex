% !TEX root = ../../Diploma.tex
\section{Wind Turbines}
\subsection{Description of a Turbine and Flow Features}
First a few basic definitions of the model used for wind turbines will be given. Today's mainstream wind turbines consist of a rotor with three blades, with a horizontal axis, therefore they are called horizontal axis wind turbines (HAWT), and the rotor points upwind.  This rotor is mounted to the nacelle, which holds the gearbox and the generator. The nacelle sits on top of the tower. The diameter of the rotor is $D$. A schematic of a HAWT is given in \autoref{fig:turbine}. It shows the wind velocity $\vec{V}$, the azimuthal angle $\Theta$ and the angular velocity $\omega$ of the rotor , the pitch angle of one of the blades $\Phi_0$, the yaw angle $\Psi$ of the turbine and the torque due to aerodynamic forces and the generator, $M_{aero}$ and $M_{gen}$, respectively.
\begin{figure}[h]
	\centering
	\def\svgwidth{0.5 \textwidth}
	\input{pics/diagrams/turbine.pdf_tex}
	\caption{Schematic of a HAWT and relevant angles, lengths, velocities and moments.}
	\label{fig:turbine}
\end{figure}\\
The flow regime a wind turbine is subjected to is the atmospheric boundary layer (ABL), which features wind velocities of the order of $\mathcal{O}(\SI{1}{m/s})$ to $\mathcal{O}(\SI{10}{m/s})$ \cite[p. 5]{kaimal_atmospheric_1994}. Furthermore the flow is turbulent and sheared. For further information on the ABL and the challenges in modelling it, the reader is referred to \cite{kaimal_atmospheric_1994} and \cite{holtslag_stable_2013}. Due to the nature of the ABL the wind speed is higher at larger hub heights, and a turbine with a larger diameter can produce more power since the available power $P_{avail}$ is proportional to the area $A$ swept by the rotor. Advances in materials science and turbine design have therefore lead to a steady increase in turbine sizes and nominal capacity \cite{rohrig_powering_2019}. Since data from real world turbines is not publicly available, the academic world uses theoretical reference turbines such as the NREL 5 MW Wind Turbine \cite{jonkman_definition_2009} or the recently proposed IEA Wind 15-Megawatt Offshore Reference Wind Turbine \cite{gaertner_iea_2020} for simulations.\cite{hansen_aerodynamics_2008} \\
A turbine induces a wake downstream, characterized by an increase of turbulence intensity and a decrease of average velocity by $\SI{50}{}$ percent and more \cite{abkar_wake_2016}. This poses a problem in wind farms, where the wake of the upstream turbines decreases the inflow velocity of the downstream turbines and thus the generated power. More details on wakes can be found in \cite{boersma_tutorial_2017}.

\subsection{Wind Turbine Control}
\label{sec:WTC}
To generate power from the aerodynamic torque acting on the rotor a generator applies a countering torque. Furthermore, the turbine is equipped with different actuators, which can be used to manipulate the behaviour in order to achieve certain goals. The operating conditions of wind turbines can be classified into three regions, depending on the wind speed. In region I, below the cut-in speed, no power is generated and the wind is used to speed up the rotor. In between cut-in and rated speed lays region II, where the goal is to maximize the generated power. At rated speed, the turbine generates the maximum power. Above rated speed, in region III, the control is focussed on the quality of the generated electricity and to minimize loads on the turbine. An example of a detailed control curve can be found in \cite{jonkman_definition_2009}. \cite{boersma_tutorial_2017} \\
The controllable variables are $\Psi$, $\Phi$ and $M_{gen}$. To maximize the power, the yaw has to be adjusted so the turbine points upwind. The pitch and generator torque have to be adjusted so that $M_{gen}$ and $\omega$ are maximized, since the generated power $P_{gen}$ is:
\begin{equation}
	P_{gen} = \omega M_{gen}.
\end{equation} The blades are designed so that the optimal blade pitch is zero. At a given wind speed and constant blade pitch, it can be shown that $M_{aero} \propto \omega^2$. Applying the law of conservation of angular momentum to the rotor and generator, with $I$ being the moment of inertia of rotor and generator, yields:
\begin{equation}
	I\frac{\mathrm{d}\omega}{\mathrm{d}t} = M_{aero} - M_{gen}. \label{eq:ang_mom}
\end{equation}
Therefore, controlling the torque in region II according to
\begin{equation}
	M_{gen} = \kappa \omega^2 \label{eq:control}
\end{equation} maximizes $P_{gen}$, with $\kappa$ being a proportionality constant that can be found experimentally. This control mechanism will be referred to as greedy control, since it seeks to maximize generated power of a single turbine. \cite[p.63 - 77]{hansen_aerodynamics_2008}
\subsection{Blade Element / Momentum Theory}
To increase efficiency of wind turbines and parks as well as model their lifetime, it is necessary to study the  the physical phenomena connected to wind turbines.To do so, models of the turbines and the airflow have been developed. Blade Element / Momentum Theory was developed to analyze the loads on a rotor. It combines one-dimensional momentum theory and local forces on a section of a blade, the so called blade element. Momentum theory assumes the flow to be steady, inviscid, incompressible and axisymmetric. The rotor is assumed to have an infinite number of blades and thus is equivalent to a permeable disc. It is based on the integral forms of conservation of mass, momenta and energy:
\begin{align}
	\oint_{\partial V} \rho \vec{u} \cdot \vec{n} \, \mathrm{d}A &= 0 \label{eq:mass_cons}\\
	\oint_{\partial V} \rho u_x \vec{u} \cdot \vec{n} \, \mathrm{d}A &= T - \oint_{\partial V} p \vec{n}\cdot \vec{e}_x \, \mathrm{d}A   \label{eq:thrust} \\
	\oint_{\partial V} \rho r u_\theta \vec{u} \cdot \vec{n} \, \mathrm{d}A  &= M_{aero} \label{eq:torque}\\
	\oint_{\partial V} \left(p + \frac{1}{2} \rho \Vert\vec{u} \Vert^2\right) \vec{u} \cdot \vec{n} \, \mathrm{d}A  &= P. \label{eq:power}
\end{align}	The surface of the control volume $V$ is denoted as $\partial V$, density is denoted as $\rho$, the velocity vector is $\vec{u}= \left[u_x, u_r, u_\theta \right]$  and $\vec{n}$ is the normal vector pointing outwards of the control volume. $T$ is the thrust force acting on the rotor in streamwise direction and $P$ the power extracted by the rotor.
The three dimensionless quantities tip speed ratio $\lambda$, thrust coefficient $C_T$ and power coefficient $C_P$ are defined as follows:
\begin{align}
\lambda &= \frac{\omega R}{V_0} \\
C_T &= \frac{T}{\frac{1}{2} \rho A V_0^2} \\
C_P &= \frac{P}{\frac{1}{2} \rho A V_0^3},
\end{align}
with $R$ being the radius of the rotor and and $V_0$ is the mean wind speed.
Applying these equations to a control volume enclosed by the stream tube around the rotor disc under the assumption that $u_x = u_r = const$ in the rotor disc, yields these basic relationships for the thrust and power:
\begin{align}
	T &= 2 \rho A V_0^2 a(1-a), \quad C_T = 4a(1-a) \label{eq:thrust2}\\
	P &= 2 \rho A V_0^3 a(1-a)^2, \quad C_P = 4a(1-a)^2 \label{eq:power2}.
\end{align}
The axial interference factor $a$ is a measure for the influence of the rotor disc on the velocity in the stream tube and defined as
\begin{equation}
	a = 1 - \frac{u_r}{V_0}.
\end{equation}
From \eqref{eq:power2} it is possible to find a maximum power coefficient, which is given by
\begin{equation}
		C_{P,max} = \frac{16}{27} \approx 59.3 \%, \quad a = \frac{1}{3} \label{eq:betz-limit},
\end{equation}
and is referred to as the Betz-Joukowski limit. In practice this limit can not be reached due to higher dimensional effects such as the rotation of the wake. Sørenensen gives an assessment of the assumptions made in \cite{sorensen_general_2016}. \cite[p. 7 - 11]{sorensen_general_2016} \\
The local forces acting on the blade are calculated assuming a 2D flow around an blade element, which has the form of an airfoil. The forces and velocities in the local coordinate system of the blade are shown in \autoref{fig:airfoil}, with $x$ being the global streamwise direction. The forces $\vec{F}_n$ and $\vec{F}_t$ are the forces on the blade element in normal and tangential direction, respectively, while the lift and drag forces are $\vec{F}_L$ and $\vec{F}_D$. The undisturbed wind speed is $V_0$, $\omega r$ is the velocity due to the rotation of the rotor and the induced velocity is defined as $\vec{V}_i = [-a V_0, a'\omega r]$, with axial and tangential induction factor $a$ and $a'$. The sum of these velocities is the relative velocity $\vec{V}_{rel}$.
\begin{figure}[H]
	\centering
	\def\svgwidth{0.5 \textwidth}
	\input{pics/diagrams/airfoil.pdf_tex}
	\caption{Schematic of the local forces and velocities at a blade element.}
	\label{fig:airfoil}
\end{figure}
The thrust and torque on the rotor within an infinitesimally thick stream tube are calculated as follows:
\begin{align}
\frac{\mathrm{d}T}{\mathrm{d}r} &=N_b F_n = \frac{1}{2}N_b \rho L_c \Vert \vec{V}_{rel} \Vert^2 C_n \label{eq:thrust_airfoil}\\
\frac{\mathrm{d}M_{aero}}{\mathrm{d}r} &= N_b r F_t = \frac{1}{2} N_b\rho L_c \Vert r \vec{V}_{rel} \Vert^2 C_t, \label{eq:torque_airfoil}
\end{align} with $N_b$ being the number of blades and $L_c$ the chord length of the airfoil. The force coefficients $C_n$ and $C_t$ can be found by projecting lift and drag of the airfoil, which can usually be found as tabulated values, into the global coordinate system. Lift and drag depend on the local angle of attack $\alpha$ and the Reynolds number. The angle of attack $\alpha$ can be found through the difference of the angle between rotor plane and $\vec{V}_{rel}$, denoted as $\phi$ and the local pitch $\Phi_l$. Combining \eqref{eq:thrust_airfoil} and \eqref{eq:torque_airfoil} with the thrust and torque found by applying momentum theory to the same stream tube yields:
\begin{align}
a &= \frac{1}{4 \sin(\phi)^2 / \left(s C_n \right) +1} \label{eq:BEM1} \\
a' &= \frac{1}{4 \sin(\phi) \cos(\phi) / \left( s C_t \right) -1}, \label{eq:BEM2}
\end{align}
with the solidity $s = N_b c/(2 \pi r)$. This allows for an iterative computation of the infinitesimal torque and thrust as described in \autoref{alg:BEM}. Integrating these with respect to the radius yields $M_{aero}$ and $T$. In practice, the blade is defined as finite sized blade elements and the integration becomes a summation. \cite[p. 100 - 103]{sorensen_general_2016} \\
\begin{algorithm}
	\caption{Algorithm to compute thrust and torque on a blade element}
	\label{alg:BEM}
	\begin{algorithmic}
		\FORALL{elements}
			\STATE $a \gets 0$
			\STATE $a'\gets 0$
			\REPEAT
				\STATE $\phi \gets \tan^{-1}((1-a)/(\lambda r/R (1+a') ))$
				\STATE $\alpha \gets \phi - \Phi_l$
				\STATE compute $C_n$ and $C_t$ from lift and drag tables
				\STATE calculate new $a$ and $a'$ from \eqref{eq:BEM1} and \eqref{eq:BEM2}
			\UNTIL{$a$ and $a'$ converge}
		\ENDFOR
		\STATE calculate $\int \mathrm{d}T$ and $\int \mathrm{d}M$ according to \eqref{eq:thrust_airfoil} and \eqref{eq:torque_airfoil}
	\end{algorithmic}
\end{algorithm}
Many corrections for BEM have been found to increase the accuracy of the model. Among them is the Prandtl tip loss factor, which proposes a factor to correct for the error due to the assumption of an infinite number of blades. Furthermore, above an axial induction factor of $1/3$, the assumptions made by BEM become inaccurate, as momentum theory predicts an expansion of the wake that is significantly too large. A correction for the calculation of $C_t$ has been proposed by Glauert. Many other corrections for effects caused by the wake have also been proposed, such as a coupled near and far wake model by Pirrung et al. \cite{pirrung_coupled_2016}. \cite[p. 103 - 104]{sorensen_general_2016}
\subsection{Actuator Line Model}
To study the interaction of fluid and turbine in detail, the flow field has to be resolved by means of computational fluid dynamics (CFD). Since fully resolving the shape of the blades would require a prohibitively fine resolution, the influence of the blade on the fluid is modelled. An overview over the models applied can be found in \cite{breton_survey_2017} and \cite{kheirabadi_quantitative_2019}. One such model is the actuator line model (ALM), which models each blade as a line of forces on the fluid. Like in BEM, the forces on the blade are calculated from local velocities and airfoil data at the points $r_j$ along the $i$th actuator line $\vec{e}_i$. These forces are then distributed by applying a convolution with a Gaussian filter kernel $\eta(d)$:
\begin{align}
	\eta(d) &= \frac{1}{\epsilon^2 \pi^{3/2}} e^{(-(d/\epsilon)^2)} \label{eq:gauss_alm} \\
	\vec{F(x)} &= \sum_{i=1}^{N_b} \int_{0}^{R} \left(\vec{F}_n(r)+ \vec{F}_t(r) \right) \eta(\Vert \vec{x} - r\vec{e}_i\Vert)\, \mathrm{d}r. \label{eq:ALM}
\end{align} The parameter $\epsilon$ is referred to as the smearing width, since it controls the stretching of the bell curve. As shown for example by Asmuth et. al \cite{asmuth_actuator_2019}, the ALM does not account for velocity induced by the root and tip vortices. This leads to an overprediction of forces on the blade, most significantly of the tangential force near the tip. Among others, Meyer-Forsting et al. have proposed corrections  based on an iterative correction of the relative velocity \cite{meyer_forsting_vortex-based_2019}. \cite{sorensen_numerical_2002}
