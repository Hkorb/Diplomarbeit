\section{Lattice Boltzmann Method}
\subsection{basics}
To conduct the training of the agent, it needs to interact with the environment. In the case studied in this thesis the environment is a wind farm with multiple turbines. Therefore the fluid field within the wind park has to be calculated. The traditional approach would be to discretize the Navier-Stokes-Equations (NSE), however, solvers based on the NSE require many CPU-hours. A newer approach is the Lattice Boltzmann Method. The basics of its theory will be layed out in this chapter. \\ 
The Lattice Boltzmann Method (LBM) is based on a discretization of the Boltzmann Equation derived from the kinetic theory of gases. This description is often referred to as a mesoscopic description, since it stands in  between the scale of continuum theory and the scale of single particles, which will be referred to as macroscopic and microscopic scale respectively. While the NSE describe the medium through density $\rho$, velocity $\vec{u}$ and pressure $p$, the Boltzmann Equation considers the particle density function (pdf) $f$, which describes the distribution of particles in six dimensional phase space. It is therefore a function of space $\vec{x}$, microscopic velocity $\vec{\xi}$ and time. However, the macroscopic quantities can easily be recovered from the pdf by taking its zeroth and first moments in velocity space, as shown in \eqref{eq:rho} and \eqref{eq:velocity}. Similarly the energy can be found by the third moment. The pressure can not be recovered directly, instead it is calculated by a state equation such as the ideal gas law. The pdf tends towards an equilibrium, which is given by the Maxwell equilibrium. It can be described by only the macroscopic quantities and is shown in \eqref{eq:equlibrium}. \cite[p. 15- 21]{kruger_lattice_2017} \\
The Boltzmann Equation describes the change of the pdf over time. The change in time is governed by the collision operator $\Omega$. It describes the change of $f$ due to collisions of particles. In the Boltzmann Equaiton it is an integral operator, that accounts for all possible collisions. Therefore it is mathematically rather cumbersome which renders it unuseful for numerical discretization, thus in LBM another formulation for $\Omega$ is used, which is an active area of research in the LBM community\cite{coreixas_comprehensive_2019}. The Boltzmann Equation is shown in \eqref{eq:boltzmann}. The second term in the second form is a convection term, while the third term corresponds to a forcing term. In comparison to the NSE the Boltzmann Equation lacks a diffusive term. Diffusion occurs only through the collision operator. \cite[p. ]{kruger_lattice_2017}
\begin{align}
	\rho(\vec{x}, t) &= \int_{-\infty}^{\infty} f(\vec{x}, \vec{\xi}, t) \mathrm{d}\vec{\xi} \label{eq:rho} \\
	\rho(\vec{x}, t) \vec{u}(\vec{x}, t) &= \int_{-\infty}^{\infty} \vec{\xi} f(\vec{x}, \vec{\xi}, t) \mathrm{d}\vec{\xi} \label{eq:velocity}\\
	f^{eq} (\vec{x}, \vec{\xi}, t) &= \rho\left(\frac{1}{2 \pi RT}\right) e^{-\Vert \vec{v} \Vert^2/(2RT) } \label{eq:equilibrium} \\
	\frac{D f(\vec{x}, \vec{\xi}, t)}{Dt} &= \frac{ \partial f(\vec{x}, \vec{\xi}, t)}{\partial t} + \frac{ \partial f(\vec{x}, \vec{\xi}, t)}{\partial\vec{x}} \frac{\partial \vec{x}}{\partial t} + \frac{ \partial f(\vec{x}, \vec{\xi}, t)}{\partial \vec{\xi}} \frac{\partial \vec{\xi}}{\partial t} =  \Omega(f) \label{eq:boltzmann}
\end{align}
In order to use the Boltzmann Equation for numerical simulations it needs to be discretized in space and time and a collision operator has to be defined.

\subsection{boundary conditions}