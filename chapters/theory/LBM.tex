\section{The Lattice Boltzmann Method}
\subsection{basics}
To conduct the training of the agent, it needs to interact with the environment. In the case studied in this thesis the environment is a wind farm with multiple turbines. Therefore the fluid field within the wind park has to be calculated. The traditional approach would be to discretize the Navier-Stokes-Equations (NSE), however, solvers based on the NSE require many CPU-hours. A newer approach is the Lattice Boltzmann Method. The basics of its theory will be layed out in this chapter. \\ 
The Lattice Boltzmann Method (LBM) is based on a discretization of the Boltzmann Equation derived from the kinetic theory of gases. This description is often referred to as a mesoscopic description, since it stands in  between the scale of continuum theory and the scale of single particles, which will be referred to as macroscopic and microscopic scale respectively. While the NSE describe the medium through density $\rho$, velocity $\vec{u}$ and pressure $p$, the Boltzmann Equation considers the particle density function (pdf) $f$, which describes the distribution of particles in six dimensional phase space. It is therefore a function of space $\vec{x}$, microscopic velocity $\vec{\xi}$ and time. However, the macroscopic quantities can easily be recovered from the pdf by taking its zeroth and first moments in velocity space, as shown in \eqref{eq:rho} and \eqref{eq:velocity}. Similarly the energy can be found by the third moment. The pressure can not be recovered directly, instead it is calculated by a state equation such as the ideal gas law. The pdf tends towards an equilibrium, which is given by the Maxwell equilibrium. It can be described by only the macroscopic quantities and is shown in \eqref{eq:equilibrium}, with $R$ being the specific gas constant and $T$ the temperature. The velocity $\vec{v}$ is defined as $\vec{v} = \vec{\xi}-\vec{u}$.\cite[p. 15- 21]{kruger_lattice_2017} \\
The Boltzmann Equation describes the change of the pdf over time. The change in time is governed by the collision operator $\Omega$. It describes the change of $f$ due to collisions of particles. In the Boltzmann Equaiton it is an integral operator, that accounts for all possible collisions. Therefore it is mathematically rather cumbersome which renders it unuseful for numerical discretization, thus in LBM another formulation for $\Omega$ is used, which is an active area of research in the LBM community \cite{coreixas_comprehensive_2019}. The Boltzmann Equation is shown in \eqref{eq:boltzmann}. The first form is the total derivative of $f$ with repsect to time. The second term in the second form is a convection term, while the third term corresponds to a forcing term. In comparison to the NSE the Boltzmann Equation lacks a diffusive term. Diffusion occurs only through the collision operator. \cite[p. 21]{kruger_lattice_2017}
\begin{align}
	\rho(\vec{x}, t) &= \int_{-\infty}^{\infty} f(\vec{x}, \vec{\xi}, t) \mathrm{d}\vec{\xi} \label{eq:rho} \\
	\rho(\vec{x}, t) \vec{u}(\vec{x}, t) &= \int_{-\infty}^{\infty} \vec{\xi} f(\vec{x}, \vec{\xi}, t) \mathrm{d}\vec{\xi} \label{eq:velocity}\\
	f^{eq} (\vec{x}, \vec{\xi}, t) &= \rho\left(\frac{1}{2 \pi RT}\right) e^{-\Vert \vec{v} \Vert^2/(2RT) } \label{eq:equilibrium} \\
	\frac{D f(\vec{x}, \vec{\xi}, t)}{Dt} &= \frac{ \partial f(\vec{x}, \vec{\xi}, t)}{\partial t} + \frac{ \partial f(\vec{x}, \vec{\xi}, t)}{\partial\vec{x}} \frac{\partial \vec{x}}{\partial t} + \frac{ \partial f(\vec{x}, \vec{\xi}, t)}{\partial \vec{\xi}} \frac{\partial \vec{\xi}}{\partial t} =  \Omega(f) \label{eq:boltzmann}
\end{align}
In order to use the Boltzmann Equation for numerical simulations it needs to be discretized in space, velocity and time and a collision operator has to be defined. \\
The velocity discretization is based on the Hermite Series expansion, since its generating function is of the same form as the equilibrium distribution. To recover the first three moments of the distribution correctly, i.e. density, velocity and energy, truncation of the series after the first three terms is sufficient and the roots of the Hermite polynomials up to order three are the necessary discrete velocities $\vec{\xi}_i$. Together with the weights $w_i$, that are used in the Gauss-Hermite quadrature, the velocities form a velocity set. Velocity sets are denoted in the DdQq-notation, with d being the number of spatial dimensions and q the number of discrete velocities. In three dimensions D3Q15, D3Q19 and D3Q27 can be used to recover the Navier-Stokes-Equations, but only D3Q27 is suitable for high Reynolds number flows \cite{kang_effect_2013}.\cite[p. 73-93]{kruger_lattice_2017} \\
Time is discretized via the explicit Euler forward scheme, which can be shown to be second order accurate. Space is discretized uniformly into a cubic grid, so that discrete pdfs move from one node of the grid to the other in one timestep. The ratio of timestep to lattice width is called the lattice speed of sound $c_s$. The timestep is usually separated into two parts, the streaming step, in which populations are advected from one node to the other, and the collision step, in which the collision operator is applied. Applying the entire discretization gives the Lattice Boltzmann Equation (LBE), which is given in \eqref{eq:LBE}, with $\vec{c}_i$ being $\vec{\xi}_i/\sqrt{3}$ for convenience. It is, compared to discretized NSE, very simple and the separation into collision and streaming makes it easily parallelizible. Furthermore the equations to compute density and velocity are given in \eqref{eq:density_d} and \eqref{eq:vel_d}. \cite[p. 94-98]{kruger_lattice_2017}\\
\begin{align}
	f_i(\vec{x} + \vec{c}_i \Delta t, \vec{c}_i, t+\Delta t ) &= \Omega(f)_i + f_i(\vec{x}_i, \vec{c}_i, t) \label{eq:LBE} \\
	\rho(\vec{x}, t) &= \sum_i f_i(\vec{x}, t) \label{eq:density_d} \\
	\rho(\vec{x}, t) \vec{u}(\vec{x}, t) &= \sum_i \vec{c}_i f_i(\vec{x}, t) \label{eq:vel_d}
\end{align}
Lastly a collision operator has to be found. The first applicable collision operator proposed was the Bhatnager-Gross-Krook (BGK) operator. It is based on the fact, that the distributions tend to the equilibrium distribution, therefore the BGK operator relaxes the distributions towards equilibrium with a constant relaxation frequency $\omega$. This leads to \eqref{eq:BGK}. Via Chapmann Enskog analysis this relaxation frequency is related to the kinetic viscosity $\nu$ by \eqref{eq:nu}. \cite[p. 98-100, 112]{kruger_lattice_2017}
\begin{align}
	\Omega^{BGK}_i = \omega\left(f_i - f_i^{eq} \right) \label{eq:BGK} \\
	\nu = c_s^2\left(\frac{1}{\omega} - \frac{\Delta t }{2} \right) \label{eq:nu}
\end{align}
While the BGK operator is sufficient for low Reynolds number flows, it becomes unstable at higher Reynolds numbers. Therefore more sophisticated methods had to be developed. One approach is to transform the distributions into moment space and relaxe the moments independently, leading to multiple relaxation time (MRT) methods. Geier et al. argue, that, since these moments are not statistically independent, they cannot be relaxed separately. However, cumulants of a distribution are statistically independent by design, therefore they can also be relaxed independently. Furthermore, they fulfill Galilean invariance, which is not always the case for MRT methods. To transform the discrete distributions into cumulant space, the discrete distributions are redefined in terms of continuous velocity and then transformed into frequency space for Galilean invariance. From the transformed distributions the cumulants are generated. These quantities can now be relaxed independently. It can be shown that with a parametrization this method can be fourth order accurate\cite{geier_fourth_2018}. However in underresolved flows, this parametrization is used to add numerical diffusivity, acting like an implicit subgrid scale model for Large-Eddy-Simulations (LES). However, the exact behaviour of the underresolved cumulant operator is not yet known. More details on the cumulant LBM as well as a derivation for a refinement algorithm can be found in appendix \ref{app:cumulant}. \\
While LBM has some advantages over NSE-based algorithms, the treatment of boundary conditions is one of its weaknesses, as usually it is not straight forward, since it is necessary to prescribe the populations in the boundary nodes. No-slip and full slip boundaries are imposed by bounce-back and bounce forward, respectively and velocity boundary conditions can be imposed by prescribing the corresponding equilibrium distributions or by extending the bounce back approach \cite[p. 175 - 189, 199 - 207]{kruger_lattice_2017}. Another problem often arising in LBM is the reflection of acoustic waves, especially at inlet and outlet boundaries. There exist methods to cancel these waves, while another approach is to simply increase the viscosity in a so called sponge layer near the outlet \cite[p. 522 - 526]{kruger_lattice_2017}.