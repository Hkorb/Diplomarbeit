% !TEX root = ../../Diploma.tex
\section{Setup}
\subsection{Preliminary Studies in the BEM environment}
To validate the implementation of the controllers and do studies of some of the hyperparameters of the agent the BEM environment is used. The airfoil data and physical properties are taken from the NREL5 reference \cite{jonkman_definition_2009}. The mean wind speed is $V_0=\SI{10.5}{m/s}$ and a velocity probe is placed ten and a half meters upstream of the turbine.
The control variable is $M_{gen}$. The state of the environment consists of the angular velocity of the turbine and the streamwise velocity of the velocity probe. Training is conducted in six environments in parallel, with independently generated turbulent inflow. The size of the timestep of the simulation is $\Delta t = \SI{0.1}{s}$.
A parameter study of the following parameters is conducted:
\begin{itemize}
	\item variance of action $\sigma_A$
	\item policy $l_2$ regularization coefficient $\varepsilon_{p,l_2}$
	\item value $l_2$ regularization coefficient $\varepsilon_{v,l_2}$
\end{itemize}
These parameters were chosen since they do not directly depend on the physics and layout of the problem. The parameters not varied are set according to \autoref{tab:params_agent}.
\begin{table}[h]
	\centering
	\caption{Parameters of agent and environment}
	\begin{tabular}{|c|c|c|}
		\hline
		Parameter & Symbol & Value \\
		\hline \hline
		Time per action & & $\SI{1.0}{s}$ \\ \hline
		Number of actions per episode & $N_{A,e}$ & $500$ \\ \hline
		Number of episodes per batch & $N_{e,b}$ & $12$ \\ \hline
		Width of actor network & & $100$ \\ \hline
		Width of value network & & $200$ \\ \hline
		Variance of action & $\sigma_A$ & $0.1$ \\ \hline
		Importance ratio clipping & $\beta$ & $0.2$ \\ \hline
		Discount factor & $\gamma$ & $0.95$ \\ \hline
		Policy $l_2$ regularizaion coefficient & $\varepsilon_{p,l_2} $ & $0.1$ \\ \hline
		Value $l_2$ regularization coefficient & $\varepsilon_{v,l_2}$ & $0.1$ \\ \hline
		Value network loss coefficient & $\varepsilon_{v,l}$ & $0.5$ \\ \hline
		Learning rate & $\psi$ & $10^{-3}$ \\
		\hline
	\end{tabular}
	\label{tab:params_agent}
\end{table}
\subsection{The LBM-ALM Environment}
\label{ssec:LBM_ALM_env}
The properties of the turbine are again taken from the NREL5 reference \cite{jonkman_definition_2009}. The diameter $D$ of the rotor is therefore $D=\SI{126}{m}$. Three turbines are placed in the center of the cross-stream plane with a distance of five diameters in streamwise direction. The first turbine is placed three diameters downstream of the inlet. The domain is $19$ diameters long and six diameters wide and high. In order to reduce computational cost, a more refined domain is placed within the first domain. It is placed in the center of the crosswise plane and one diameter downstream of the inlet. The inflow is turbulent. The parameters of the domain are gathered in \autoref{tab:params_domain}. \\
\begin{table}[h]
	\centering
	\caption{Parameters of the domain}
	\begin{tabular}{|c|c|}
	\hline
	Parameter & Value \\ \hline \hline
	Rotor diameter & \SI{126}{m} \\ \hline
	Outer domain H x W x L & $\SI{6}{D} \times \SI{6}{D} \times \SI{19}{D}$ \\ \hline
	Outer domain resolution & $\SI{8}{nodes/D}$ \\ \hline
	Inner domain H x W x L & $\SI{5}{D} \times \SI{5}{D} \times \SI{16}{D}$ \\ \hline
	Inner domain resolution & $\SI{16}{nodes/D}$ \\ \hline
	Turbulence intensity & $\SI{5}{\percent} $ \\ \hline
	Mean wind speed & $\SI{10.5}{m/s} $ \\ \hline
	\end{tabular}
	\label{tab:params_domain}
\end{table}
Training of the agents is again conducted in six environments in parallel.
\subsection{Optimizing Specific Behaviour}

\subsection{Learning New Behaviour}
